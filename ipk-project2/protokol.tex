%Encoding: utf8
%Author: Pavol Loffay, xloffa00@stud.fit.vutbr.cz
%Date: 3.3.2012
%Project: Popis protokolu pre projekt 2 do predmetu IPK

\documentclass[10pt,a4paper]{article}

%preambule
\usepackage[slovak]{babel}
\usepackage[top=2.5cm, left=1.5cm, text={18cm, 25cm}]{geometry}
\usepackage[IL2]{fontenc}
\usepackage[utf8]{inputenc}
\newcommand\czuv[1]{\quotedblbase #1\textquotedblleft}

%textova cast
\begin{document}

\title{Počítačová komunikácia a siete\,--\,projekt 2. \\ Aplikačný protokol}
\author{Pavol Loffay\\xloffa00@stud.fit.vutbr.cz}
\date{\today}
\maketitle

\section{Úvod}
Tento dokument popisuje aplikačný protokol, ktorý som navrhol pre prenos 
informácii medzi serverom a klientom.

\section{Popis protokolu}
Podľa zadania projektu bolo nutné, aby klient požadoval od serveru
preklad domény na IPv4, alebo IPv6 adresu. Pričom mohol požiadať o obe adresy
naraz. Zo strany serveru bolo nutné implementovať aby poslal IPv4, IPv6 alebo
obe adresy súčasne. Ak sa nepodaril preklad, bolo nutné poslať túto informáciu
klientovi.

\subsection{Značky posielané klientom}
Požiadavky klienta som sa rozhodol transformovať na správu,
\texttt{GET\_IPV4}, ak klient žiada o IPv4 adresu. Alebo \texttt{GET\_IPV6}, ak
klient požiada o IPv6 adresu, alebo \texttt{GET\_IPIP}, ak požiada o obe súčasne.
Za týmito značkami vždy nasleduje
\texttt{<značka><medzera><doména><medzera><\textbackslash r\textbackslash
n\textbackslash r\textbackslash n>}. 

\begin{table}[h]
    \begin{center}
        \begin{tabular}{lc}
        Žiada preklad na adresu IPv4 & \texttt{GET\_IPV4} \\
        Žiada preklad na adresu IPv6 & \texttt{GET\_IPV6} \\
        Žiada preklad na adresy IPv6 a IPv4 & \texttt{GET\_IPIP}
        \end{tabular}
        \caption{Značky posielané klientom}
    \end{center}
\end{table}

\subsection{Značky posielané serverom}
Server odpovedá klientovi značkou \texttt{IPV4}, ak našiel preklad na IPv4
adresu. Ďalej značkou \texttt{IPV6}, ak našiel preklad na IPv6 adresu. Za oboma
týmito značkami nasleduje medzera a príslušná adresa
\texttt{<značka><medzera><IP><medzera>}. Ak sa nepodaril preklad 
na ani jednu adresu, o ktorú požadoval klient, server odpovie správou
\texttt{NOT\_FOUND}.

\begin{table}[h]
    \begin{center}
        \begin{tabular}{ll}
        Adresa IPv4 & \texttt{IPV4} \\
        Adresa IPv6 & \texttt{IPV6} \\
        Preklad domény na IP adresu sa nenašiel & \texttt{NOT\_FOUND}
        \end{tabular}
        \caption{Značky posielané serverom}
    \end{center}
\end{table}
\end{document}

